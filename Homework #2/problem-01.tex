% !TEX root = main.tex

\section{Задача №1. Проверка параметрических гипотез}

\paragraph{Условие.} Для исследования стабильности температуры в термостате с кварцевым генератором были проведены две серии замеров температуры (в $C^{\circ}$) с интервалов в 15 часов:
\begin{align*}
\vec{X} = (17.85, 17.98, 18.01, 18.2, 17.9, 18.0), \\
\vec{Y} = (18.01, 17.98, 18.05, 17.9, 18.0).
\end{align*}

Считая распределение контролируемого признака нормальным со среднеквадратичным отклонением $\sigma=0.1~C^{\circ}$, при уровне значимости $\alpha=0.05$ проверить гипотезу о неизменности температуры в термостате.



\paragraph{Решение.}\hfill\\
Согласно условию, 
\begin{align*}
X \sim N(m_1, \sigma_1^2),  \\
Y \sim N(m_2, \sigma_2^2),
\end{align*}
причем $\sigma_1=\sigma_2=0.1$ и $m_1=\Expect X$, $m_2=\Expect Y$ неизвестны.

Введём основную гипотезу:
\begin{align*}
H_0 = \{температура~в~термостате~\emph{не~изменилась}\} = \{m_1 = m_2\}.
\end{align*}
С учётом выборочных средних $\bar{X}=17.99$ и $\bar{Y}=17.988$ естественно ввести такую конкурирующую гипотезу:
\begin{align*}
H_1 = \{температура~в~термостате~\emph{уменьшилась}\} = \{m_1 > m_2\}.
\end{align*}
Используя статистику
\begin{align*}
T(\vec{X}, \vec{Y}) = \frac{\bar{X} -\bar{Y}}{\sqrt{(\frac{\sigma_1^2}{n_1} + \frac{\sigma_2^2}{n_2})}}, 
\end{align*}
где $n_1, n_2$ - размеры выборок и $T(\vec{X}, \vec{Y})\sim N(0, 1)$ при истинности гинотезы $H_0$, построим критическое множество: 
\begin{align*}
W=\{(\vec{x}, \vec{y}): T(\vec{x}, \vec{y})  \geq u_{1-\alpha}\}, 
\end{align*}
где $u_{1-\alpha}$ - квантиль нормального распределения уровня $1-\alpha=0.95$.
Вычислим статистику $T(\vec{x}, \vec{y})$:
\begin{align*}
T(\vec{x}, \vec{y}) = \frac{17.99 - 17.988}{\sqrt{(\frac{0.01}{6} + \frac{0.01}{5})}} = \frac{0.002}{0.06} \approx 0.04.
\end{align*}
При $u_0.95=1.645$
\begin{align*}
0.04 \ngeq 1.645 \Rightarrow (\vec{x}, \vec{y}) \notin W \Rightarrow \\
 \Rightarrow ~принимаем~H_0,~отклоняем~H_1.
\end{align*}

\paragraph{Ответ.} Температура в термостате \emph{не изменилась}.