% !TEX root = main.tex

\section{Задача №1. Предельные теоремы теории вероятностей}

\paragraph{Условие.} За значение некоторой величины принимают среднее арифметическое $500$ измерений. Предполагая, что среднее квадратичное отклонение возможных результатов каждого измерения не превосходит $0.5$, оценить вероятность того, что отклонение найденного таким образом значения величины от истинного не превосходит $0.2$.

\paragraph{Решение.}\hfill\\
Предположим, что истинное значение величины равно $X_{real}.$ Тогда измеренное значение величины
 \[
 	X = \frac{1}{n}\sum_{i=1}^n X_{i}, где n=500. 
\]
\\
Обозначим также отклонение от истинного за $\varepsilon$, дисперсию каждого измерения за 
\[
	\Variance X_{i}=\sigma^2=0.5^2=0.25.
\]
\\
Необходимо найти 
\[
    \Prob\{|X-X_{real}| < \varepsilon\} =  \Prob \Bigl\{-\varepsilon < \frac{X_{1}+ ... + X_{n} - nX_{real}}{n} < \varepsilon \Bigl\} = 
 \]
\[
    = \Prob\Bigl\{--\varepsilon  \sqrt{\frac{n}{\sigma^2}} < \frac{X_{1}+ ... + X_{n} - nX_{real}}{\sqrt{n\sigma^2}} < \varepsilon  \sqrt{\frac{n}{\sigma^2}}\Bigl\} .
 \]
 \\
 Так как $n >> 1$, то мы можем использовать \emph{Центральную предельную теорему}, согласно которой величина $\frac{X_{1}+ ... + X_{n} - nX_{real}}{\sqrt{n\sigma^2}}$ распределена по нормальному закону. Тогда
 \[
	\Prob\{|X-X_{real}| < \varepsilon\} \approx \Phi \left(\varepsilon \sqrt{\frac{n}{\sigma^2}}\right) - \Phi \left(- \varepsilon \sqrt{\frac{n}{\sigma^2}}\right) =
	2\Phi_{o} \left(\varepsilon \sqrt{\frac{n}{\sigma^2}}\right) = 
 \]
 
 \[
  = 2\Phi_{o} \left(0.2 \sqrt{\frac{500}{0.25}}\right) = 2\Phi_{o} \left( 8.9 \right)   \approx 1
 \]
 		
 



\noindent

\paragraph{Ответ.} $\Prob\{|X-X_{real}| < 0.2\}\approx1$.