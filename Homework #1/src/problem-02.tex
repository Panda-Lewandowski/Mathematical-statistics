% !TEX root = main.tex

\section{Задача №2. Метод моментов}

\paragraph{Условие.}  
С использованием метода моментов для случайной выборки $\vec{x}_n = (x_1, \dots, x_n)$ найти точечные оценки параметров заданного закона распределения генеральной совокупности $X$.

\[
f_{X}(x)= \frac{1}{4!\theta^5}x^4 e^{-x/\theta},~x > 0.
\]

\paragraph{Решение.} Рассморим данный закон распределения. Если заметить, что $\Gamma(5)=4!$, то нетрудно понять, что $f_{X}(x)$ есть гамма-распределение  с параметрами $\lambda=\theta,~\alpha=5$.

\[
f_{X}(x)=x^4\frac{e^{-x/\theta}}{\theta^5 \Gamma(5)},~x > 0.
\]

Отсюда следует, что $X \sim \Gamma(5, \theta)$. Поэтому начальный момент 

\[
m_1(\theta) = \Expect X=5\theta.
\]

Начальный выборочный момент
\[
\hat{\mu_1} = \bar{X},
\]

где $\bar{X}$ -- среднее выборочное.

Следуя методу моментов и приравнивая эти моменты между собой получаем 
\[
5\theta=\bar{X},
\theta=\frac{\bar{X}}{5}.
\]

\paragraph{Ответ.} $\theta=\frac{\bar{X}}{5}$, при  $x>0$.