% !TEX root = main.tex

\section{Задача №3. Метод максимального правдоподобия}

\paragraph{Условие.} С использование метода максимального правдоподобия для случайной выборки $\vec{X} = (X_1, \dots, X_n)$ из генеральной совокупности $X$ найти точечные оценки параметров заданного закона распределения. Вычислить выборочные значения найденных оценок для выборки $\vec{x}_n = (x_1, \dots, x_5)$.

\[
f_{X}(x)= \frac{\theta^3}{2!}x^2 e^{-\theta x},~x > 0;
\]

\[
\vec{x}_5 = (0.4, 1.5, 0.8, 0.7, 2.7).
\]

\paragraph{Решение.}
\noindent
Как и в предыдущей задаче, мы имеем дело с гамма-распределением  с параметрами $\lambda= \frac{1}{\theta},~\alpha=3$.

\[
f_{X}(x)=x^2\frac{e^{-\theta x}\theta^3}{\Gamma(3)},~x > 0.
\]

Отсюда следует, что $X \sim \Gamma(3, \frac{1}{\theta})$.

Определим функцию правдоподобия:

\[
L(\vec{X_n}; \theta) =  \frac{\theta^{3n}}{\Gamma(3)^n}\prod_{i = 1}^{n} \left(X_i^2 e^{-\theta X_i}\right) \eqno(1)
\]

Пролагарифмируем (1):

\[
\ln L(\vec{x_n}; \theta) = \ln \left( \frac{\theta^{3n}}{\Gamma(3)^n}\right) + \ln \left(\prod_{i = 1}^{n} \left(x_i^2 e^{-\theta x_i}\right)\right) = 3n\ln\theta - n\ln2 + \ln \left(e^{-\theta \sum_{i = 1}^n x_i} \prod_{i = 1}^{n} x_i^2\right) = 
\]
\[
= 3n\ln\theta - n\ln2 + \ln \left(e^{-\theta \sum_{i = 1}^n x_i} \right) + \ln \prod_{i = 1}^{n} x_i^2 =  3n\ln\theta - n\ln2 + \theta \sum_{i = 1}^n x_i  + \ln \prod_{i = 1}^{n} x_i^2.  \eqno(2)
\]

Продифференцируем (2) и приравняем к нулю:

\[
\frac{\partial \ln L(\vec{x_n}; \theta)}{\partial \theta} = 0;
\]

\[
\frac{\partial \ln L(\vec{x_n}; \theta)}{\partial \theta} = \frac{3n}{\theta} - \sum_{i = 1}^n x_i = 0.
\]

Отсюда 

\[
\theta = \frac{3n}{\sum_{i = 1}^n x_i}.
\]

Для  $\vec{x}_5 = (0.4, 1.5, 0.8, 0.7, 2.7)$ получаем:

\[
\theta = \frac{3 * 5}{0.4 + 1.5 + 0.8 + 0.7 + 2.7} \approx 2.46.
\]

\paragraph{Ответ.} $\theta \approx 2.46$.